\documentclass[print]{friggeri-cv_cn} % Add 'print' as an option into the square bracket to remove colors from this template for printing

\addbibresource{bibliography.bib} % Specify the bibliography file to include publications
\linespread{1.2}
\usepackage{hyperref}
\usepackage{xcolor}
\definecolor{darkblue}{rgb}{0.0,0.5,1.0}
\hypersetup{
    colorlinks,
    linkcolor={darkblue},
    citecolor={darkblue},
    urlcolor={darkblue}
}

\begin{document}

\header{陆}{哲琪}{浙江大学计算机辅助设计与图形学国家重点实验室} % Your name and current job title/field

%----------------------------------------------------------------------------------------
%	SIDEBAR SECTION
%----------------------------------------------------------------------------------------

\begin{aside} % In the aside, each new line forces a line break
\section{个人信息}
男,24周岁
毕业时间:2017-3
(+86)17816861269
\href{mailto:albuscrow@gmail.com}{albuscrow@gmail.com}
\section{教育经历}
本科:重庆大学计算机学院
硕士:浙江大学计算机学院
\section{作品链接}
%\href{http://app.mi.com/detail/81435}{会小秘}
\href{http://www.hisensehitachi.com/jstx/index.aspx?nodeid=1162}{iZE智能控制器Android端}
\href{https://github.com/albuscrow/AFFD}{AFFD三维模型编辑器}
\href{http://www.jinxizi.cn/android/huozhanggui.html}{二维火掌柜Android端}
\href{https://github.com/albuscrow/learn_unity}{雷电游戏}
\href{http://36kr.com/p/5043478.html}{哇陶Android端}
\href{https://github.com/albuscrow}{更多...}
\section{编程语言}
基本掌握:Java,C/C++
较为熟练:Python,Lisp Matlab,C\#,JavaScript
大致了解:SQL,Haskell
%{\color{red} $\varheartsuit$} Java 熟练掌握
\section{各类技能}
\LARGE{Android}\normalsize{,}\LARGE{Linux}
\large{OpenGL/GLSL}\normalsize{,}\large{CUDA}
\large{Git}\normalsize{,}\large{Vim}\normalsize{,GCC,Cmake}
\normalsize{QT,Docker,Emacs,Latex}
\small{Struts}\normalsize{,}\small{Spring}\normalsize{,}\small{Hibernate}
\small{Django}\normalsize{,}\small{JQuery}\normalsize{,}\small{unity3d}
\section{兴趣爱好}
\normalsize{折腾各种系统和软件
学习新的语言和框架
看书,跑步,玩游戏}
\end{aside}
%----------------------------------------------------------------------------------------
%	WORK EXPERIENCE SECTION
%----------------------------------------------------------------------------------------
\section{项目经历}
\begin{entrylist}
%------------------------------------------------

\entry
{三维模型编辑器(Python,C++,Qt,OpenGL,CUDA)}
{2015}
{这是硕士阶段的科研任务,目标是实现一个灵活、鲁棒、高效的三维模型局部编辑算法。
    收获:
\begin{itemize}
    %\item 主体算法由\href{http://link.springer.com/article/10.1007/s11766-014-3239-6}{AFFD}算法改进而来,修改了其中切割三角形的部分
    %\item 通过OpenGL compute shader,实现GPU并行加速和跨平台, 基本掌握GLSL;
    %\item 极大的增强的自己的数学功底,对并行环境下的程序开发有了一定的了解;

    \item 基本掌握GPU并行环境下的程序开发;
    \item 极大的增强的自己的数学功底;
\end{itemize}}
\\
\entry
{哇陶(Java,Android)}
{2015}
{哇陶是所在实验室与\href{http://wowtao.me/index.html}{哇陶科技}合作开发的一款软件,致力于将DIY陶瓷的过程从线下带
    到线上,并借此打造一个瓷器交易平台。我负责Android端的开发。收获:
\begin{itemize}
    \item 一定程度上扮演了项目经理的角色,具备一定的项目管理经验;
    \item 尝试解决需求变化和代码质量、工程进度之间的矛盾。在此过程中为解决此类问
        题积累了一定的经验;
\end{itemize}}
\\
\entry
{移动APP开发(Java,Android)}
{2014-2016}
{为多家公司(\href{http://www.hjtechcn.cn/}{华舰}、\href{http://www.2dfire.com/}{二维火}、\href{http://d-controls.com/}{迪浩斯}、\href{http://www.hisensehitachi.com/}{日立})以外包的形式开发/维护过共5款Android App,均顺利交付。收获:
\begin{itemize}
    \item 熟练掌握了Android应用开发;
%的各种要素:屏幕适配、网络请求、工程构建、
    %    协同开发、代码管理、Bug追踪、测试、模块化开发、各类开发工具的使用等;
    \item 意识到了大型工程中的沟通成本和文档的重要性;
    \item 报酬约为6k-10k/月;
\end{itemize}}

\end{entrylist}

\section{各类奖项}

\begin{entrylist}
%------------------------------------------------
\entryac
{美国数学建模竞赛特等奖提名(食物在烤箱中的受热分析及优化)}
{2013}
\entryac
{重庆大学第二届“树声前锋杯”学生创新创业大赛二等奖(365食名志)}
{2013}
\entryac
{重庆大学大学生科研训练计划一等奖(漫画分割算法)}
{2013}
\entryac
{国家奖学金,国家励志奖学金等各类奖学金}
{2010-2014}
%National Encouragement scholarship
%\entryac
%{重庆大学创新创业先进个人}
%{2013}

%------------------------------------------------
\end{entrylist}
\sectionend{社会实践}
\begin{entrylist}
%------------------------------------------------
\entry
{大学生创业}
{2012}
{和同学合伙创立\href{http://nineton.cn/}{重庆九吨科技有限公司},以技术入股。负责技术开发,期间主要工作:
\begin{itemize}
\item 合作完成一个类似于大众点评的餐饮平台------\href{http://nineton.cn/product.php?id=17}{365食名志};
\item 独立完成一款零食购买APP------\href{http://apk.gfan.com/Product/App584080.html}{果果零食};
\end{itemize}
}
%------------------------------------------------
\\
\entryend
{AutoDesk上海研发中心实习}
{2016}
{探索DirectX3 12的新特性,并探索将其用于AutoCAD以提高该软件绘图性能的可行性:
\begin{itemize}
    \item 对现代GPU的特性有了更加深入的了解;
    \item 对大型软件开发的流程有了更加全面的认识;
\end{itemize}
}
\end{entrylist}
\end{document}
%2013 美国数学建模竞赛特等奖提名(食物在烤箱中的受热分析及优化)
%2013 重庆大学第二届“树声前锋杯”学生创新创业大赛二等奖(365食名志)
%2013 重庆大学大学生科研训练计划一等奖(漫画分割算法)
%2010-2014 获得国家奖学金,国家励志奖学金等各类奖学金
%2011 重庆大学寒假招生宣传先进个人
%我目前最喜欢的就是写代码,准确的说是让一个东西从无到有或者从不能work到work的过程,我在这个过程中能获得极大的满足感。
%我曾经编写过很多软件,当然也会遇到很多问题,有的卡了一两天的问题在解决的一刹那,那种畅快的感觉很难用语言形容。
%我已经决定将来要成为一名程序员,干这一行就要有活到老学到老的觉悟。我很享受学习的过程,希望能在学习中不断的进步。
%
%印象最深的一个项目就是到目前都没有完成的哇陶项目,这是浙大cadcg实验室和南京哇套科技有限公司合作的一个DIY陶瓷的项目。
%我负责Android端的程序,项目涉及到的东西并不复杂,就是OpenGL和一些常规的安卓开发技术。
%之所以令我印象深刻是因为这个项目让我在现实世界中经历的软件工程课堂所描述的各种典型问题:
%1、项目需求不断变化,甚至到现在都在改变。这会导致代码难以维护,以及无法按时交付。
%2、项目没有专业的项目经历,程序,美工,需求方之间的交流沟通成本十分高。
%3、项目并没有前面的需求分析,设计阶段,直接进入编码阶段。
%4、项目并没有很好的文档化,很多交流都在微信群完成。
%
%这些问题都是课堂上提到过的,但只有到真正面对这些问题的时候,才会意识到这些问题的严重性和必要性(比如我以前觉得没必要写文档)。虽然这个项目不算成功,但是却让我见识到了很多东西,也学到了很多东西。相信以后如果自己有机会管理一个团队,有了这个教训一定能少走很多弯路。
